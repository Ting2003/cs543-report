\section{Tile Connection Graph}
After tile extraction and matching step, we need to build tile connection graph. Each vertex of the graph is a tile, and the edges in the 
graph represent different type of regions. An example graph is shown in Figure~\ref{eg_graph}:

\begin{figure}[htbp]
	  \centering
	  \includegraphics[width=0.5\textwidth]{example_graph.pdf}
	  \caption{An example graph}
	  \label{eg_graph}
\end{figure}

In Figure~\ref{eg_graph}, edge lines of different color represent different region type. In order to build such a graph, we first utilize the 
neighboring information of tiles in the extraction step to locate all neighbors of each tile. Here we use 8-connection structure. Then, we divide each edge of a tile into 3 segments. For 4 edges, there will be 12 line segments. Besides, we also need to represent the middle region 
of the tile, where it can be the dead end of road, city or cloister. Another item is used to represent whether there is a shield in the tile or not. As a 
result, for each tile, we create a vector of 14 dimensions to represent one tile. For each template tile, we manually create such a vector. With the tile index and rotation for each extracted tile from template matching process, we can modify the order of the items in the 
vector to generate the corresponding vector for each extracted tiles.

With the information extracted from above steps, we can create the tile connection graph. 