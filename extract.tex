\section{Tile Extraction}
After taking the pictures, we have to extract tiles from the pictures.
In this section, we are going to discuss details about the tile 
extraction process.


\begin{figure}[htbp]
	  \centering
	  \includegraphics[width=0.5\textwidth]{game.pdf}
	  \caption{Original image}
	  \label{Original_im}
\end{figure}

Figure~\ref{Original_im} shows the whole picture without follower. 
In the extraction process, the aim is to extract the boundary lines of 
tiles. Since hough transformation method is good for detecting lines, we
first extracted edge pixels with canny edge detector and tried hough 
transform to detect boundary lines of tiles. However, the result is not 
good. A lot of boundary lines of tiles are missing, and we can not recover 
the boundary lines for each tile.

We propose another method which can effectively extract the boundary lines of
 tiles. The main process is as follows:
\begin{enumerate}
	\item Extract edge map with oriented filters.
	
			Use oriented filters, we first extract and edge information of the 
			pictures. We assume all the boundary lines of tiles are along horizontal 
			and vertical direction. As a result, we only keep edge pixel whose 
			magnitude is within $5\,^{\circ}$ of horizontal and vertical direction. 
			The extracted edge map is shown in Figure~\ref{Edge_map}.
			
			\begin{figure}[htbp]
				  \centering
				  \includegraphics[width=0.5\textwidth]{oriented_edges.pdf}
				  \caption{Edge map}
				  \label{Edge_map}
			\end{figure}
			
	\item Extract boundary lines with connected component method.
	
			After extracting edge map with oriented filters, there are still a lot of 
			discontinuous edge pixels in edge map. Most of these edge pixels are within 
			the tile boundaries and is not of our interest. As a result, we filter these 
			discontinuous edge pixels with some threshold. Then, we treat the continuous 
			pixels in edge map as a component, and extract boundary line of these components.
			Again, we keep those lines which are very close to horizontal and vertical ones.
			The resulting distribution of extracted lines are shown in Figure~\ref{Connect}.
			In Figure~\ref{Connect}, the white lines are those edges after filtering step but 
			are beyond the thresholding range of horizontal and vertical lines.
			
			 \begin{figure}[htbp]
				  \centering
				  \includegraphics[width=0.5\textwidth]{short_edges.pdf}
				  \caption{Connect Component}
				  \label{Connect}
			\end{figure}
			
\end{enumerate}

 

