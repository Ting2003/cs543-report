\section{Tile Extraction}
After taking the pictures, we have to extract tiles from the pictures.
In this section, we are going to discuss details about the tile 
extraction process.


\begin{figure}[htbp]
	  \centering
	  \includegraphics[width=0.5\textwidth]{game.pdf}
	  \caption{Original image}
	  \label{Original_im}
\end{figure}

Figure~\ref{Original_im} shows the whole picture without follower. 
In the extraction process, the aim is to extract the boundary lines of 
tiles. Since hough transformation method is good for detecting lines, we
first extracted edge pixels with canny edge detector and tried hough 
transform to detect boundary lines of tiles. However, the result is not 
good. A lot of boundary lines of tiles are missing, and we can not recover 
the boundary lines for each tile.

We propose another method which can effectively extract the boundary lines of
 tiles. The main process is as follows:
\begin{enumerate}
	\item Extract edge pixels with oriented filters.
	
			Use oriented filters, we first extract and edge information of the 
			pictures. We assume all the boundary lines of tiles are along horizontal 
			and vertical direction. As a result, we only keep edge pixel whose 
			magnitude is within $5\,^{\circ}$ of horizontal and vertical direction. 
			The extracted edge map is shown in :
			
			
			\begin{figure}[htbp]
				  \centering
				  \includegraphics[width=0.5\textwidth]{oriented_edges.pdf}
				  \caption{Edge map}
				  \label{Edge_map}
			\end{figure}
	\item
\end{enumerate}

 

