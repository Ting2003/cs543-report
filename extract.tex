\section{Tile Extraction}
\label{sec:extract}
After taking the pictures, we have to extract tiles from the pictures.
In this section, we are going to discuss details about the tile 
extraction process.

\begin{figure}[htbp]
	  \centering
	  \includegraphics[width=0.5\textwidth]{game.pdf}
	  \caption{Original image}
	  \label{Original_im}
\end{figure}

Figure~\ref{Original_im} shows the whole picture without follower. In the
extraction process, the aim is to extract the boundary lines of tiles. Since
hough transform is good for detecting lines, we first extracted edge pixels with
canny edge detector and tried hough transform to detect boundary lines of tiles.
However, the result is not good. A lot of boundary lines of tiles are missing,
and we can not recover the boundary lines for each tile.

We propose another method which can effectively extract the boundary lines of
tiles. The main process is as follows: \begin{enumerate} \item Extract edge map
with oriented filters. With oriented filters, we first extract edge map of the
picture. We assume all the boundary lines of tiles are along horizontal and
vertical directions. As a result, we only keep edge pixel whose magnitude is
within $5\,^{\circ}$ tolerance of horizontal and vertical direction. The
extracted edge map is shown in Figure~\ref{Edge_map}.

\begin{figure}[htbp]
\centering
\includegraphics[width=0.5\textwidth]{oriented_edges.pdf}
\caption{Edge map}
\label{Edge_map}
\end{figure}

\item Extract boundary lines with connected component method. After extracting
edge map with oriented filters, there are still a lot of discontinuous edge
pixels in edge map. Most of these edge pixels are within the tile boundaries and
is not of our interest. We filter out these discontinuous edge pixels with some
threshold. Then, we treat the continuous pixels in edge map as a component, and
extract boundary line of these components. Again, we keep those lines which are
very close to horizontal and vertical ones. The resulting distribution of
extracted lines are shown in Figure~\ref{Connect}. In Figure~\ref{Connect}, the
white lines are those edges after filtering step but are beyond the thresholding
range of horizontal and vertical lines.

\begin{figure}[htbp] \centering
\includegraphics[width=0.5\textwidth]{short_edges.pdf}
\caption{Connect Component}
\label{Connect}
\end{figure}

\item Merge lines to generate boundary lines of tiles. In Figure~\ref{Connect},
we can see that the extracted lines covers most of the tile boundaries. Although
a few boundary lines are missing, it can be recovered by boundary lines along
the same row or column. In some region, there are several boundary lines making
a thick line region. There are also some other false short lines in
Figure~\ref{Connect}. Both of these two case need to be taken care of.

First, we extend all the lines and intersect the lines with the boundaries of
image. Then, for the thick line region, we utilized non-maximum like technique
to produce a thin line. Specifically, we search the neighborhood of each line.
If there exist some other lines nearby, we merge the neighboring lines. At the
same time, we accumulate all the number of pixels the merged lines covered in
edge map. Difference between true and false boundary lines is that the false
lines covers only a very small part of pixels in edge map, while the true lines
covers much larger number of pixels in edge map. With this observation, we set a
threshold for the number of pixels each line covered in edge map, and most of
the false lines are filtered out.

Most of the left lines are true boundary ones, while there might still be
several false lines. In order to effectively eliminate these lines, we perform
histogram of distance along every two neighboring lines of horizontal and
vertical directions. The higher the distance, the higher weight for the distance
will be. The peak of the histogram represents the horizontal and vertical side
length of each tile. We then start from the first line from horizontal and
vertical direction, and incrementally add the voted side length. In order to
avoid the accumulate error, we round the added results to the true line
position. The results of merging lines is shown in Figure~\ref{Merge_line}.

\begin{figure}[htbp] \centering
\includegraphics[width=0.5\textwidth]{clustered_edges.pdf}
\caption{Merge lines}
\label{Merge_line}
\end{figure} 
\end{enumerate} 

The white pixels in Figure~\ref{Merge_line} is the edge map by Canny. From
Figure~\ref{Merge_line}, it can be seen that the extracted lines is in highly
correspondence with the boundary lines of edges. Assume the intersection of lines
are a set of spots. It can been seen that there are spots contain no tile, and
spots contain tile. We observed that most of the space of false spots are clean,
while in the edge area of the spots, there might be some noise generated in the
tile extraction process. On the other hand, edge pixels within true spots are
more evenly distributed within all regions of the spots. As a result, we set a
threshold for the number of pixels in middle region and only keep the spots which
contains bigger number of pixels than the threshold. Experimental results showed
that this method is very effective and is able to extract only the true tiles.
The extracted tiles are listed in Figure~\ref{ex_tiles}.


\begin{figure}[htbp]
\centering
\includegraphics[width=1.0\textwidth]{no_follower_tiles.pdf} 
\caption{Extracted tiles}
\label{ex_tiles}
\end{figure}