\section{Follower Detection}
\label{sec:detect}
To score for each player, we first need to know the position of their follower.
Our main task in this step is to detect whether if there is a follower in 
a given query image, and report the color and bounding box of the follower
to the next step, which determines the region this follower is occupying.
We consider the following several approaches:
\begin{itemize}
\item 
We train a classifier to recognize follower in an image.
This is probably a robust approach, but requires much more effort and
need to have training data.
\item
We compare two images, one with follower and one without follower.
This seems be an sound approach at a first glance. However, due to the noise, lighting condition and misalignment of these two images extracted from the first step, we cannot simply subtract and compare the difference.
\item
Since the follower have distinct color, we may use it as a cue and use color thresholding. We preset the value of five different colors, and
use it to filter the RGB channel such that the strongest response to the filter
is retained. We first implement this approach, and it works well for some
followers with distinct colors such as blue.
However, the threshold is not robust to lighting condition.
\end{itemize}

We adopted an approach that combines the ideas of the last two. To utilize the
information between two tile images (one with follower and one without), we use
RANSAC to fit a homography to warp the two images. The rational behind is that
the size of the follower is relatively small. The interest points it occludes
should be a small portion of all of the interest points we can detect. Thus, the
probability that we compute a correct homography is high. Next, we subtract two
images and post-process it by removing the small regions. By doing so, we can
robustly find the difference between two images, which is exactly the follower we
are looking for. We can then use color to determine what player it belongs to.
Finally, we will send the bounding Although the running time is comparatively
longer, it is still within seconds for an images.
Figure~\ref{fig:follower} shows three examples generated by this
method, where the left figure shows the query image and the right one
shows the detected follower in bitmap image with bounding box.
\begin{figure}[hbt]
	\centering
\subfigure[]{
\includegraphics[width=.14\textwidth]{fo5.jpg}
}
\subfigure[]{
\includegraphics[width=.14\textwidth]{dt5.jpg}
}
\subfigure[]{
\includegraphics[width=.14\textwidth]{fo14.jpg}
}
\subfigure[]{
\includegraphics[width=.14\textwidth]{dt14.jpg}
}
\subfigure[]{
\includegraphics[width=.14\textwidth]{fo49.jpg}
}
\subfigure[]{
\includegraphics[width=.14\textwidth]{dt49.jpg}
}
	\caption{Example output of follower detection}
	\label{fig:follower}
\end{figure}