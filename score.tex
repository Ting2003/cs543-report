\section{Compute Score of Followers}
In this section, we will use the example graph from Figure~\ref{eg_graph} to illustrate the computing score process.

\subsection{Find out connected regions}
As we can see from Figure~\ref{eg_graph}, there are several connected regions, such as field, city and road. We need to find out how many 
connected regions of field, city and road respectively. 

We first find total number of regions in each tile, and assign each line segment into different regions. Then, all the tiles have a list 
of regions. An example of finding regions for one tile is shown in Figure~\ref{tile_region}. The number indicates the region index each 
line segments belongs to.

\begin{figure}[htbp]
	  \centering
	  \includegraphics[width=0.5\textwidth]{tile_region.pdf}
	  \caption{Regions in one tile}
	  \label{tile_region}
\end{figure}

After locating all regions for each of the extracted tile, we need to merge the regions of each tile in the tile connection graph and extract 
all connected regions. After merging all the regions from all tiles, we obtain a region list where each of the edge segment is assigned to 
the corresponding region. These regions are classified into field, road and city.

\subsection{Compute score for each connected regions}
In this section, we compute the scores for each connected region including cloister, field, city and road. 

It is defined that the scores of region such as city equals to the number of tiles 
contained in the connected region. If there is a shield in some tile, the score of this region is added by the number of tiles containing 
shield in the region. For connected region such as road, the score equals to number of tiles in the region. Since each region contains the tiles included in the region, with the rules for city and road regions, it is easy to compute the scores.

For connected field regions, it is complicated to find out the score. The score of a field is not correlated to the number of tiles within the region but is proportional to the number of completed city regions within the field. As a result, we need to first identify the completed 
city regions in the graph, and find out which field regions each completed city region belongs to.

We detect the complete city by check whether all the edge segments in the connected region have a neighbor. If all the edge segments in the 
region have neighbors, than this region is complete, or else is not. To find out which field regions this complete city belongs to, for each 
line segment belongs to the region, we find out its neighboring field line segment and assign the complete city to the field region the line segment belongs to.

For cloister, after seeing there is a cloister, we check number of neighboring tiles around the tile containing cloister. Assume there are $N(N<=8)$ tiles around cloister, the score for the cloister region is $N+1$.

\subsection{Compute score for each follower}