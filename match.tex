\section{Tile Matching}
\label{sec:matching}
After extracting the tile images, we will perform matching such that the regions
corresponding to each tile can be identified. Our task is to match about 70 query
tiles to about 25 template tiles. This step is based on interest point matching,
where the features we used is SIFT descriptor~\cite{lowe1999object}. SIFT
descriptor is a type of local descriptor that is scale invariant and proven to be
very robust and reliable. The basic idea of SIFT matching is to extract SIFT
local descriptors from two images, use RANSAC to compute a homography and try to
find the match points. However, this will be slow since we have lots of images
and each image will contain hundreds of SIFT descriptors. Since the template
tiles are known to us, we can pre-compute the SIFT descriptors and store them in
files as `SIFT database'. Given an input image, we extract its SIFT descriptors
and try to find a nearest neighbor from the database for each of them. Then, we
can use a major vote to vote for a matching template since the match will have
lots of matched keypoints with high probability. One problem with the naive
implementation is that the computational cost for searching a nearest neighbor
will be high. Assume that there are $n$ input images and $m$ templates, where
each of them contains to $O(k)$ SIFT local descriptors. Then, to find all nearest
neighbors for the one SIFT descriptor, we need to compare it with $O(mk)$
descriptors in the database, and there will $O(nmk^2)$ comparisons. To overcome
this problem, we can use KD-tree to speed up the query. Then, each search will
require only $O(\log (mk))$ time. This improves the performance dramatically.
Figure~\ref{fig:sift} illustrates the flow.
Figure~\ref{fig:sift:kdtree} shows that given an query image,
we finds nearest neighbors for all of the SIFT descriptors of it
in the KD-tree, where the SIFT descriptors are overlaid with the query
images. Figure~\ref{fig:sift:template} shows the template image 
that has the most votes and the matched SIFT descriptors.

Note that after finding a match, we still need to find the rotation of the query.
Traditional approach will be using RANSAC to fit a rotation matrix. In fact,
there are only four possible rotations in our case. Thus, we can simply rotate
the query image four times ($0^\circ$, $90^\circ$, $180^\circ$, $270^\circ$).
For each rotation, we only need to compute the sum of square distance between
the matched keypoints and find the smallest one. 

\begin{figure}[b]
\centering
\subfigure[]{
\includegraphics[height=.13\textheight]{sift.jpg}
\label{fig:sift:kdtree}
}
\hfill
\subfigure[]{
\includegraphics[height=.13\textheight]{keypoints.jpg}
\label{fig:sift:template}
}
\caption{SIFT matching.}
\label{fig:sift}
\end{figure}