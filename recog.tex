\section{Follower Occupied Region Recognition}
After detecting the position of follower, we need to recognize the type of region that the follower is occupied. Considering the types of 
regions, including "field", "city", and "road" has obvious difference in color, we decide to use color intensity as the feature for 
recognizing the occupied region of each follower. For the fourth type of region, which is "cloister", since all its boundary edge in the tile 
is field or at most 1 road segment, it will be individually taken care of and is beyond this category.

We noticed that the color intensity of different regions along blue channel as well as the difference of green and red channel has obvious 
difference, which is shown as Figure~\ref{Channel_diff}.

    \begin{figure}[htbp]
        \centering
        \subfigure[][Channel color intensity of City]{
                \includegraphics[width=0.2\textwidth]{city.pdf}\label{subfig.1}}
        \subfigure[][Channel color intensity of Field]{
                \includegraphics[width=0.2\textwidth]{field.pdf}\label{subfig.2}}
	\subfigure[][Channel color intensity of Road]{
	         \includegraphics[width=0.2\textwidth]{road.pdf}\label{subfig.3}}

        \caption{Channel color intensity of different regions}
        \subref{subfig.1} Channel color intensity of City
        \subref{subfig.2} Channel color intensity of Field
		\subref{subfig.3} Channel color intensity of Road
        \label{Channel_diff}
     \end{figure}

With the color intensity difference along different channels, we set some threshold for histogram of color intensities along the 
2 channels. By the difference in results of different regions, we successfully recognize the region each follower is occupied. The 
results are in Figure~\ref{recog_results}.

	\begin{figure}[htbp]
		  \centering
		  \includegraphics[width=0.9\textwidth]{recog_results.pdf}
		  \caption{Recognition results of occupied region of each follower}
		  \label{recog_results}
	\end{figure}

 
