\section{Follower Occupied Region Recognition}
After detecting the position of follower, we need to recognize the type of region each follower occupies. Considering the types of 
regions, including, ``field", ``city", and``road" are very different in color, we decide to use color intensity as the feature for 
recognizing process. There is one more type of region, which is ``cloister". Since almost all of its boundary edges in the tile 
are fields, it will be individually taken care of and we are not discuss this case in this session.

We noticed that the color intensity of different regions along blue channel as well as the difference of green and red channel has obvious 
difference, which is shown as Figure~\ref{channel_diff}. In Figure~\ref{channel_diff}, ``Blue" means image from blue channel, while ``Green - Red" means image from the difference of green and red channel. 

	\begin{figure}[htbp]
		  \centering
		  \includegraphics[width=0.9\textwidth]{channel_diff.pdf}
		  \caption{Channel color intensity of different regions}
		  \label{channel_diff}
	\end{figure}

With the color intensity difference along different channels, we set some threshold for histogram of color intensities along the 
2 channels, and successfully recognize the region each follower is occupied. The 
results are in Figure~\ref{recog_results}.

	\begin{figure}[htbp]
		  \centering
		  \includegraphics[width=0.9\textwidth]{recog_results.pdf}
		  \caption{Recognition results of occupied region for each follower}
		  \label{recog_results}
	\end{figure}

 
